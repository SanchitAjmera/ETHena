\documentclass[a4paper12pt, twocolumn]{article}
\usepackage{geometry}
\usepackage[]{graphicx}
\usepackage{authoraftertitle}
\graphicspath{ {./images/} }



\title{\includegraphics[scale=0.5]{lunoLogo.png}\\ \vspace{1.5cm}--Team Name-- Luno Trading Challenge Report}

% \author{
%   Sanchit Ajmera\\
%   Department of Computing\\
%   Imperial College London\\
%   sanchitajmera2017@gmail.com
%   \and 
%   Luqman Liaquat\\
%   Department of Computing,\\
%   Imperial College London\\
%   luqman.liaquat90@gmail.com
%   \and
%   Manuj Mishra\\
%   Department of Computing,\\
%   Imperial College London\\
%   manujmishra2000@gmail.com
%   \and
%   Shivam Patel\\
%   Department of Mathematics,\\
%   Imperial College London\\
%   shivpatel1306@gmail.com
%   \and
%   Devam Savjani\\
%   Department of Computing,\\
%   Imperial College London\\
%   devamsavjani@rocketmail.com
% }

\author{\\
  \author{}Sanchit Ajmera\\
         \small{\affaddr{Department of Computing,}}\\
         \small{\affaddr{Imperial College London}} \\
         \small{\email{sanchitajmera2017@gmail.com}}
  \and \\
  \author{}Luqman Liaquat\\
         \small{\affaddr{Department of Computing,}}\\
         \small{\affaddr{Imperial College London}} \\
         \small{\email{luqman.liaquat90@gmail.com}}
  \and\\
  \author{}Manuj Mishra\\
         \small{\affaddr{Department of Computing,}}\\
         \small{\affaddr{Imperial College London}} \\
         \small{\email{manujmishra2000@gmail.com}}
  \and\\
  \author{}Shivam Patel\\
         \small{\affaddr{Department of Mathematics,}}\\
         \small{\affaddr{Imperial College London}} \\
         \small{\email{shivpatel1306@gmail.com}}
  \and\\
  \author{}Devam Savjani\\
         \small{\affaddr{Department of Computing,}}\\
         \small{\affaddr{Imperial College London}} \\
         \small{\email{devamsavjani@rocketmail.com}}
}

\begin{document}
\begin{titlepage}
       \maketitle
\end{titlepage}

\section{Introduction}
We are a group of first year students at Imperial College London studying Computing and Mathematics. We all have had an interest in trading and how we can implement algorithms in order to make useful and reliable predictions about the market and because of this we as a group were attracted to this hackathon.
\\We as individuals took up different roles, where some of the group researched and implemented strategies, back tested the strategies and used the server in order to use the bots for the whole day.

\section{Motivation}

\section{Scope and Overview}

\section{Project flow}
We had initially split the project into 4 stages. The first being researching the way the crypto-market works, what affects the market and different approaches we could take. The next stage was implementing the strategies and our research. We then back tested the implementations using historical data to see how much profit we would make if the bot is used to trade live. The last stage was using the bot to live trade and putting the bot on our server to run 24 hours a day.
\section{Implementation}
In order to learn, gather as much information as possible and make a good predictive measure we researched numerous strategies and trading rules.
\subsection{Moving Average Convergence Divergence}
This was the first strategy we implemented after reviewing multiple strategies and to familiarise ourselves with the Luno API. The strategy signals to buy when a shorter term moving average crosses over a longer term moving average and signals to sells when the opposite occurs.
\subsection{Relative Strength Index}
The Relative Strength Index (RSI) is an indicator, which is intended to chart the current and historical strength or weakness of a stock or market based on the closing prices of a recent trading period.
%   \subsection{Moving Stoploss}
%   The Moving Stoploss strategy is used to determine when to sell. As the price rises the stop loss will rise with it and then if the price then falls the bot will then sell out.
\subsection{Offset}
The Offset strategy buys when the price  significantly drops in comparison to a moving average and sells when the price significantly rises in comparison to the moving average.
\subsection{Candlestick Analysis}
Candlestick Analysis involves finding a variety of patterns based on the open, high, low and close price of a trading period. These include 123 Reversal, Hammer, Inverse Hammer, Three White Slaves and Morningstar trends.
\subsection{Risk Management}
After creating bots which traded based on the strategies mentioned we created a confidence marker which determined the volume which we should trade.
\subsection{Summary of Strategies}
Our strategies were mostly successful however the RSI bot made the most profit, without using the risk management strategy so that became our primary bot.

\section{Structure of Codebase}
\end{document}